\section{Problem Statement}
\label{sec:Problem-Statement}

As mentioned in section \ref{sec:Related-Work}, stress is one of the main causes for serious health concerns.
Because of this, there are stress monitors that supervise blood volume, pulse, emotions, and 
body temperature. Stress and anxiety can occur at any given time. In particular, during computer usage, stress levels 
can fluctuate as a result of the content or tasks the user is seeing or doing. Using the WESAD dataset, two popular 
classifiers, Support Vector Machines (SVM) and K-Nearest Neighbor (kNN), are used to help analyze the dataset 
to detect when and why a test subject became stressed during the course of the experiment. 
A comparison is then done to compare the performance of utilizing both Central Processing Units (CPUs) and 
Graphical Processing Units (GPUs) to perfrom this task. 
Finally, a detailed analysis of benchmarks for each test is provided to determine which method and platform 
performed the best. Below is a table denoting each testing platorm that will be used for this experiment. 

\begin{table}[h!]
\centering
\caption{Testing Platforms and thier Specifications}
\label{table:testing-platforms}
\begin{tabular}{||c c c||} 
 \hline
 Identifier & Platform  & Specifications  	\\ [0.5ex] 
 \hline\hline
 A 	& HP Spectre x360 	 & 8th Gen i7, Dedicated NNvidia GPU 	\\ 
 B 	& Custom Desktop	 & One GPU   	\\
 C 	& Custom Desktop	 & Three GPUs	\\
 D 	& Custom Desktop	 & 6 GPUs	\\
 E 	& Sabine Cluster	 & 5704 CPU Cores, 12 GPU Nodes 	 	\\ [1ex] 
 \hline
\end{tabular}
\end{table}

Note that in table \ref{table:testing-platforms}, platforms B through D will use the same Nvidia 1080 Ti GPUs and 
Celeron 6th generation CPU.
