\section{Problem Statement}
\label{sec:Problem-Statement}

Stress is one of the main causes for serious health diseases like heart attacks. 
Because of this, there are stress monitors which supervise  blood volume, pulse, emotions, and 
body temperature. Stress and anxiety can occur in any given time. In particular, during computer usage, stress levels 
can fluctuate as a result of the content or tasks the user is seeing or doing. Using the WESAD dataset, two popular 
classifiers, Support Vector Machines (SVM) and K-Nearest Neighbor (kNN), will be used to help analyze the dataset 
to detect when and why the test subject became stressed. 
A comparison is then done to compare the performance of utilizing both Central Processing Units (CPUs) and 
Graphical Processing Units (GPUs) to perfrom this task. 
Finally, a detailed analysis of benchmarks for each test is provided to determine which method and platform 
performed the best. Below is a table denoting each testing platorm that will be used for this experiment. 

\begin{table}[h!]
\centering
\caption{Testing Platforms and thier Specifications}
\label{table:testing-platforms}
\begin{tabular}{||c c c||} 
 \hline
 Identifier & Platform  & Specifications  	\\ [0.5ex] 
 \hline\hline
 A 	& HP Spectre x360 	 & 8th Gen i7, Dedicated NNvidia GPU 	\\ 
 B 	& Custom Desktop	 & One GPU*   	\\
 C 	& Custom Desktop	 & Three GPUs*	\\
 D 	& Custom Desktop	 & 6 GPUs*	\\
 E 	& Sabine Cluster	 & 5704 CPU Cores, 12 GPU Nodes 	 	\\ [1ex] 
 \hline
\footnote{* All GPUs for Platforms B through D will be Nvidia 1080 Ti Cards}
\end{tabular}
\end{table}
