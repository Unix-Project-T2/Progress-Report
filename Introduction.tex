\section{Introduction}
\label{sec:Introduction}

In recent years, the trend of applying Machine Learing (ML) concepts to improve decision making, predciting 
market sentiment, recognizing patterns, understanding consumers/patients, and interpreting text. 
Within the realm of ML concepts are classification problems where the objective is to classify data entries 
as either categorical, ordinal, or binary. 
As a result, classification algorithms (CAs) were developed using statistical analysis techniques. 

When applied to a dataset, CAs can provide great insight into common features, known as class,
found within the data. Classifiers in particular use the training data to assess how the data's respective 
attributes fit within the definitions of a particular class \cite{class}. In this paper, the use of two 
popular classifiers, Support Vector Machines (SVM) and K-Nearest Neighbor (kNN), are used to help 
analyze the Wearable Stress and Addect Detection (WESAD) dataset \cite{dataset}. A comparison is then done 
to compare the performance of utilizing Central Processing Units (CPUs) and Graphical Processing Units (GPUs). 
